\Section{Оценки под SETH}{2 сентября}

\subsection{Введение}

\textit{Зачем нужно Fine-grained complexity? Мы отвратительны в доказательстве нижних оценок. Поэтому мы делаем следующее: берём задачу, которую очень долго не могут решить, рассматриваем её как гипотезу и в этом предположении строим условные нижние оценки на другие задачи.}

\begin{problem}($k$-SAT)\\
   Дана формула на $n$ логических переменных в КНФ, такая что размер каждого клоза не больше $k$. Проверить, существует ли означивание переменных, выполняющее формулу.
\end{problem}

\begin{hypothesis}(ETH)~\cite{Impagliazzo2001}\\
   $3$-SAT не решается за время $2^{o(n)}$.
\end{hypothesis}

\begin{hypothesis}(SETH)~\cite{Impagliazzo2001}\\
   Для $\forall \epsilon > 0$ найдётся $k > 0$, такое что $k$-SAT не решается за время $2^{(1-\epsilon)n}$.
\end{hypothesis}

\begin{statement}
    SETH $\SO$ ETH.
\end{statement}

\begin{proof1}
    Сведём \s{$k$-SAT} $\to$ \s{3-SAT}, добавлением $(k - 3) m$ новых переменных. Чтобы получить линейное разрастание числа переменных, воспользуемся Sparsification леммой (\TODO: ссылка на лемму).
\end{proof1}

\begin{remark}
    Это единственные именно \textit{гипотезы}, все остальные будут \textit{conjecture}. Причина того, что эти ребята гипотезы (по словам Ивана) в том, что авторы не сильно в них верили.
\end{remark}

\begin{remark}
    Если мы сломаем \s{3-Sum}-conjecture, то просто получим более быстрый алгоритм для \s{3-Sum}. Если сломаем ETH, то перевернём мир схемной сложности (\TODO: ссылка на теорему про ETH и $E^{NP}$).
\end{remark}

\begin{definition}(Fine-grained сведение)\\
    Будем говорить, что задача $P$ $\mathbf{(T_1, T_2)}$ \textbf{fine-grained сводится} к задаче $Q$ (пишем $P \xto{T_1, T_2} Q$), если существует такой алгоритм $A$, который решает $P$ с оракульным доступом к $Q$ так что:
    \begin{itemize}
        \setlength\itemsep{-0.2em}
        \item Сложность $A$ на входе размера $n$ составляет $\O(T_1(n)^{1 - \alpha})$ для $\alpha > 0$
        \item Для $\forall \delta > 0$ найдётся $\eps > 0$, так что для любого входа размера $n$оракульные запуски $S_1, \dots S_k$ удовлетворяют следующему условию: $\sum\limits_{i = 1}^k T_2(S_i)^{1 - \delta} \le T_1(n)^{1 - \eps}$
    \end{itemize}
\end{definition}

\subsection{Нижние оценки под SETH}

Здесь и далее вместо условных нижних оценок будет писать только fg-сведения. 

\begin{problem}(\s{Orthogonal Vectors (OV)})
   Даны $2$ набора $A$ и $B$ из $n$ векторов из $\{0, 1\}^d$, где $d = o(n)$. Нужно узнать существуют ли $a \in A, b \in B$, такие что: $\sum_{i = 1}^{d} a_i b_i = 0$.
\end{problem}

\begin{reduction}(\s{$k$-SAT} $\xto{2^n, n^2}$ \s{Orthogonal Vectors})~\cite{Patrascu2010}\\
    Построим наборы размера $2^{n/2}$ и размерности $m$. В первом наборе на $i$-ой позиции поставим 0, если данное означивание первых $n/2$ переменных выполняет $i$-ый клоз. Во втором аналогично. Теперь скалярное произведение двух наборов будет равно 0 $\EQ$ данное означивание выполняет все клозы.
\end{reduction}

\begin{problem}(\s{$d$-Hitting Set})\\
   Дан универс $\cool{U}$, $|\cool{U}| = n$ и набор $\cool{S}$ подмножеств $U$ мощности не более $d$. Проверить, существует ли $X \subseteq \cool{U}$, $|X| \le k$, такой что $\forall i, S_i \cap X \neq \varnothing$.
\end{problem}

\begin{reduction}(\s{$k$-SAT} $\xto{2^n, 2^{n/2}}$ \s{$d$-Hitting Set})\\
Возьмём в качестве универса литералы (переменные и их отрицания), в качестве множеств из $\cool{S}$: $\{x_i, \overline{x_i} \}$ (чтобы выбрать означивание) и $\{ x_{i, 1}, \dots x_{i, k} \}$ (литералы, выполняющие $i$-ый клоз), получаем $d \le max(2, k)$.
\end{reduction}

\begin{reduction}(\s{$k$-SAT} $\xto{2^n, 2^n}$ \s{$d$-Hitting Set})~\cite{Cygan2016} \href{https://www.mimuw.edu.pl/~malcin/dydaktyka/2012-13/fpt/fpt_14_seth.pdf#page=2}{тык}\\
Создадим универс из $n'$ (определим позднее) элементов, которые разобьём на группы по $p$, где $2 \nmid p, p \mid n'$. Заставим брать в Hitting set ровно $\lfloor p/2 \rfloor$ элементов из каждого блока: тогда каждый блок закодирует $\binom{p}{\lfloor p/2 \rfloor}$ вариантов~--- означивание для $\alpha_p =  \lfloor \log \binom{p}{\lfloor p/2 \rfloor} \rfloor$ переменных. $\frac{\alpha_p}{p} = \frac{\lfloor \log \binom{p}{\lfloor p/2 \rfloor} \rfloor}{p} \sim \frac{\log (\frac{2^p}{\sqrt{p}})}{p} \xrightarrow[p \to \infty]{} 1$, так что размер универса будет $n' = \frac{n}{\alpha_p} p \sim n$. 

Чтобы в каждом блоке бралось хотя бы по $\lfloor p/2 \rfloor$ элементов, положим все подмножества из $\lceil p/2 \rceil$ элементов в $\cool{S}$ (теперь, если мы взяли меньше $\lfloor p/2 \rfloor$, то дополнение этих элементов не похичено). Также докинем в $\cool{S}$ все дополнения подмножеств размера $\lfloor p/2 \rfloor$, которые не соответствуют означиваниям (такие могли появиться из-за округлений). Так как $p$~--- константа, всех этим множеств будет какое-то линейное от $n$ число.

Чтобы в каждом блоке бралось не более $\lfloor p/2 \rfloor$ элементов положим $|X| = k = \frac{n}{\alpha_p} \lfloor p/2 \rfloor$.

Осталось заставить это всё выполнять клозы. Пусть клоз $c_i$ содержит литералы $x_{i_1}, \dots, x_{i_{k_i}}$ из блоков $b_{i_1}, \dots , b_{i_{k_i}}$. Тогда переберём все означивания переменных в этих блоках, не выполняющие клоз $c_i$ и положим объединение дополнения соответствующих им подмножеств размера $\lfloor p/2 \rfloor$ в $\cool{U}$. Так как $p$~--- константа, получаем линейное от $m$ число множеств.

Можем ещё оценить $d$ как $\max(\lceil p/2 \rceil, k \cdot \lfloor p/2 \rfloor)$

\end{reduction}

\subsection{Семинар}

\begin{exerc}
    Построить $(2^n, n^t)$ fg-сведение \s{$k$-SAT} $\to$ \s{$t$-Dominating Set}
\end{exerc}

\begin{exerc}
    Построить $(2^n, n^{t - 1})$ fg-сведение \s{$k$-SAT} $\to$ \s{$t$-Sparse Dominating Set} ($\frac{|E(G)|}{|V(G)|} = n^{o(1)}$)
\end{exerc}

\begin{problem}($k$-SUM)\\
   Дано $k$ массивов $A_1, \dots, A_k$ длины $n$, состоящие из целых чисел из $\{-M..M\}$, где $M = n^{\O(1)}$. Существуют ли индексы $j_1, \dots j_k$ такие что: $\sum\limits_{i = 1}^k A_{i, j_i} = 0$.
\end{problem}

\begin{exerc}
    Покажите нижнюю оценку $n^{\Omega(k)}$ для \s{$k$-Sum}, построив цепочку fg-сведений\\ \s{3-SAT} $\to$ \s{1-in-3-SAT} $\to$ \s{$k$-Sum} (в \s{1-in-3-SAT} хотим выполнить клоз ровно одной переменной)
\end{exerc}


% \begin{problem}($3$-SUM)\\
%    Дано $3$ набора $A$, $B$ и $C$ из $n$ целых чисел из $\{-M..M\}$, где $M = n^{\O(1)}$. Существуют ли $i, j$ и $k$ такие что: $a_i + b_j + c_k = 0$.
% \end{problem}
